%% Based on a TeXnicCenter-Template by Gyorgy SZEIDL.
%%%%%%%%%%%%%%%%%%%%%%%%%%%%%%%%%%%%%%%%%%%%%%%%%%%%%%%%%%%%%

%------------------------------------------------------------
%
\documentclass{amsart}
%
%----------------------------------------------------------
% This is a sample document for the AMS LaTeX Article Class
% Class options
%        -- Point size:  8pt, 9pt, 10pt (default), 11pt, 12pt
%        -- Paper size:  letterpaper(default), a4paper
%        -- Orientation: portrait(default), landscape
%        -- Print size:  oneside, twoside(default)
%        -- Quality:     final(default), draft
%        -- Title page:  notitlepage, titlepage(default)
%        -- Start chapter on left:
%                        openright(default), openany
%        -- Columns:     onecolumn(default), twocolumn
%        -- Omit extra math features:
%                        nomath
%        -- AMSfonts:    noamsfonts
%        -- PSAMSFonts  (fewer AMSfonts sizes):
%                        psamsfonts
%        -- Equation numbering:
%                        leqno(default), reqno (equation numbers are on the right side)
%        -- Equation centering:
%                        centertags(default), tbtags
%        -- Displayed equations (centered is the default):
%                        fleqn (equations start at the same distance from the right side)
%        -- Electronic journal:
%                        e-only
%------------------------------------------------------------
% For instance the command
%          \documentclass[a4paper,12pt,reqno]{amsart}
% ensures that the paper size is a4, fonts are typeset at the size 12p
% and the equation numbers are on the right side
%
\usepackage{amsmath}%
\usepackage{amsfonts}%
\usepackage{amssymb}%
\usepackage{graphicx}
%------------------------------------------------------------
% Theorem like environments
%
\newtheorem{theorem}{Theorem}
\theoremstyle{plain}
\newtheorem{acknowledgement}{Acknowledgement}
\newtheorem{algorithm}{Algorithm}
\newtheorem{axiom}{Axiom}
\newtheorem{case}{Case}
\newtheorem{claim}{Claim}
\newtheorem{conclusion}{Conclusion}
\newtheorem{condition}{Condition}
\newtheorem{conjecture}{Conjecture}
\newtheorem{corollary}{Corollary}
\newtheorem{criterion}{Criterion}
\newtheorem{definition}{Definition}
\newtheorem{example}{Example}
\newtheorem{exercise}{Exercise}
\newtheorem{lemma}{Lemma}
\newtheorem{notation}{Notation}
\newtheorem{problem}{Problem}
\newtheorem{proposition}{Proposition}
\newtheorem{remark}{Remark}
\newtheorem{solution}{Solution}
\newtheorem{summary}{Summary}
\numberwithin{equation}{section}
%--------------------------------------------------------
\begin{document}
\title[$\beta-\eta$ Model Implementation]{$\beta-\eta$ Model Implementation}
\author{P. Caspers}
\author{R. Lichters}
\email[P. Caspers]{pcaspers1973@googlemail.com}
\email[R. Lichters]{pcaspers1973@googlemail.com}
\date{May 31, 2013}
\dedicatory{First Version May 31, 2015 - This Version May 31, 2015}
\begin{abstract}
We describe the implementation of the $\beta-\eta$ model \cite{betaeta}, \cite{piterbarg}, 11.3.2.6 in QuantLib \cite{ql}.
\end{abstract}

\maketitle

\section{Model}

The driving process is given by

\begin{equation}
dx(t) = \alpha(t) ( 1 + \beta x(t) )^\eta dW(t) 
\end{equation}

with $x(0)=0$, $\alpha:\mathbb{R}\rightarrow\mathbb{R}$, $\beta > 0$ and $0 \leq \eta \leq 1$. The dynamics is expressed in the measure $\mathbb{Q}^N$ associated to the numeraire

\begin{equation}
N(t,x(t)) = e^{\lambda(t)x(t)+M(0,0;T)}
\end{equation}

with a function $\lambda: \mathbb{R}\rightarrow\mathbb{R}$ subject to constraints $\lambda(0)=0$ and $\lambda'(0)=1$. Here we define

\begin{equation}\label{formula_M}
M(t,x;T) = \log E \left( e^{-\lambda(T)(X(T)-x)} \middle | X(t)=x \right)
\end{equation}

The transition density is given in \cite{betaeta}, (4.6a), (4.6b) and (4.8). 

In the special case of $\eta=0$ we can use the closed form representation (C.3) for the density and (C.5) for $M(t,x;T)$.

\section{Tabulation of $M(t,x;T)$}

For $\eta>0$ formula \ref{formula_M} is too expensive to evaluate during a pricing. Instead it is proposed in \cite{betaeta} to write $M$ in two variables (instead of three) for each fixed $\eta$ and precompute these values. In the following we give details on how this can achieved.

The transition density is originally expressed in variables $y$ and $\tau, \overline{\tau}$ instead of $x, \overline{x}$ and $t, \overline{t}$ using the transformations (cf. \cite{betaeta}, (4.4))

\begin{equation}
y = \frac{|1+\beta x|^{1-\eta}{\beta(1-\eta)}
\end{equation}

and

\begin{equation}
\tau = \int_0^t \alpha(s) ds,
\end{equation}


\section{Parametrization of $\lambda$}

We follow \cite{piterbarg} and describe $\lambda$ by a (piecewise) constant mean reversion $\kappa: \mathbb{R}\rightarrow\mathbb{R}_{>0}$ thanks to

\begin{equation}
\lambda(t) = \frac{1 - e^{-\kappa(t) t}}{\kappa(t)}
\end{equaiton}

The normalization constraints $\lambda(0)=0, \lambda'(0)=1$ are immediately verified. Furthermore

\begin{equation}
\kappa(t) = -\frac{\lambda''(t)}{\lambda'(t)}
\end{equation}

for all $t$ where $\lambda$ is continuous. 





\begin{thebibliography}{2}

\bibitem{betaeta} Patrick S. Hagan, Diana E. Woodward, Markov interest rate models, Applied Mathematical Finance 6, 233–260 (1999)

\bibitem{piterbarg}Leif B. G. Andersen, Vladimir V. Piterbarg, Interest Rate Modeling, Atlantic Financial Press, 2010

\bibitem{ql}QuantLib A free/open-source library for quantitative finance, http://www.quantlib.org

\end{thebibliography}

\end{document}


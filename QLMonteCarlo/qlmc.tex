\documentclass{beamer}

\usetheme{CambridgeUS}
\usefonttheme{professionalfonts}

\usepackage{epstopdf}
\usepackage{graphicx}
\usepackage[miktex]{gnuplottex}
\usepackage{minted}

\usemintedstyle{manni}
\definecolor{mintedBg}{rgb}{0.98,0.98,0.70}

\begin{document}
\title{QL MC / Early Exercise}  
\author{Peter Caspers}
\institute{IKB}
\date{February 26, 2014} 

\frame{\titlepage} 

\frame{\frametitle{Table of contents}\tiny \tableofcontents[hideallsubsections]}

\section{Basics}

\begin{frame}
\frametitle{Terminology I}
\begin{itemize}
\item Exercise Value = NPV in case of exercise
\item Continuation Value = NPV in case of no exercise
\item Control Value = NPV of control variate (optional)
\item State = State variable(s) of model
\item Basis System = Functions of model's state variables
\end{itemize}
\end{frame}

\begin{frame}
\frametitle{Terminology II}
For an exercise time $i\geq 1$, Cumulated Cashflows := NPV of cashflows between $i$ and $i+1$
where $i+1$ may stand for $\infty$ and $i=0$ is reserved for the period
[evaldate, first exercise]. Here ``between'' is not referring to the payment
time but to the the cashflow belonging to the exercise into part for 
exercise $i$.
\end{frame}

\begin{frame}
\frametitle{NPV are deflated NPVs always !}
We will have to compare NPVs accross different times. This is possible
for deflated NPVs, i.e. NPVs divded by the numeraire. Deflated NPVs
are ``timeless'' (because the transformation to an absolute NPV as
of time t is done by multiplication with the numeraire N(t)).
\end{frame}

\begin{frame}
\frametitle{Exercise Strategy}
\begin{itemize}
\item Exercise if and only if Exercise Value $\geq$ Continuation Value
\item Continuation Value is not known
\item Estimate Continuation Value as a function of the Basis System
\item Exercise is never optimal, therefore option value is underestimated
\end{itemize}
\end{frame}

\begin{frame}
\frametitle{Side note: Control Variate}
Choose product $C$
\begin{itemize}
\item with high correlation to product $V$ to price
\item known analytical value
\end{itemize}
and use
\begin{equation}
E(V) = E(V+\gamma(C-E(C)))
\end{equation}
Then there is some optimal $\gamma$ which minimizes the variance of the rhs argument. C is called Control Variate. By renormalization we can assume $\gamma = -1$.
\end{frame}

\section{Exercise Strategies}

\begin{frame}
\frametitle{Basis Systems}
Typical examples include
\begin{itemize}
\item polynomials (of different type) in model's state variables or
\item in financial quantities like forward rates, swap rates
\end{itemize}
\end{frame}

\begin{frame}
\frametitle{Strategies}
\begin{itemize}
\item Longstaff Schwartz: Continuation Value = Linear function in Basis System
\item Parametric Exercise: Exercise = Arbitrary Indicator function in Basis System
\end{itemize}
\end{frame}

\begin{frame}
\frametitle{Strategy building}
Generate $m$ training paths and collect data on the training paths:
\begin{itemize}
\item cumulated cash flows
\item exercise values ($-\infty$ if ``addtional'' grid point only)
\item basis system values
\item control values
\end{itemize}
\end{frame}

\begin{frame}
\frametitle{Longstaff Schwartz}
Backward iteration through exercise times, in each step $i$:
\begin{itemize}
\item Least square linear regression of cumulated cash flows on basis system values
\item Update $i-1$ cumulated cash flows by adding $\max($ estimated continuation value, exercise value $)$
\end{itemize}
It is essential of using the estimated continuation value instead of the cumulated cashflows
on the respective path, because we are not allowed to use future's information.
Note that averaging at $i=0$ already gives the products NPV.
\end{frame}

\begin{frame}
\frametitle{Parametric exercise}
For a fixed parameter set
\begin{itemize}
\item Apply a generic exercise trigger depending on the parameters
\item Add exercise value or cumulated cashflows depending on the decision
\end{itemize}
Optimize the parameters to maximaize the mean of the realized NPV over the training paths.
\end{frame}


\section{QuantLib Support}

\begin{frame}[fragile]
\frametitle{NodeData}
Represents a ``node'' (exercise time, path). We will have number of paths times
exercise times nodes in the end.
\begin{minted}[fontsize=\footnotesize, bgcolor=mintedBg]{c++}
    struct NodeData {
        Real exerciseValue;
        Real cumulatedCashFlows;
        std::vector<Real> values;
        Real controlValue;
        bool isValid;
    };
\end{minted}
\verb+values+ are the basis system values, the remaining member variables are self
explaining.
\end{frame}

\begin{frame}[fragile]
\frametitle{genericLongstaffSchwartzRegression}
Computes least square linear regression coefficients for the basis system.
\begin{minted}[fontsize=\footnotesize, bgcolor=mintedBg]{c++}
    Real genericLongstaffSchwartzRegression(
                std::vector<std::vector<NodeData> >& simulationData,
                std::vector<std::vector<Real> >& basisCoefficients)
\end{minted}
\verb+simulationData+ should contain exercise times + 1 vectors of size $m$ = number of paths.
The first entry in simulation data corresponds to the case $i=0$ above.
The return value is the (deflated !) product NPV. The coefficients are written to the second
input parameter.
\end{frame}


\begin{frame}[fragile]
\frametitle{ParametricExercise}
Base class for arbitrary parametric exercise strategies.
\begin{minted}[fontsize=\footnotesize, bgcolor=mintedBg]{c++}
  class ParametricExercise {
    public:
    virtual ~ParametricExercise() {}
    // possibly different for each exercise
    virtual std::vector<Size> numberOfVariables() const = 0;
    virtual std::vector<Size> numberOfParameters() const = 0;
    virtual bool exercise(Size exerciseNumber,
       const std::vector<Real>& parameters,
       const std::vector<Real>& variables) const = 0;
    virtual void guess(Size exerciseNumber,
       std::vector<Real>& parameters) const = 0;
  };
\end{minted}
\end{frame}

\begin{frame}[fragile]
\frametitle{genericEarlyExerciseOptimization}
Computes an optimal parameteric exercise strategy
\begin{minted}[fontsize=\footnotesize, bgcolor=mintedBg]{c++}
    Real genericEarlyExerciseOptimization(
        std::vector<std::vector<NodeData> >& simulationData,
        const ParametricExercise& exercise,
        std::vector<std::vector<Real> >& parameters,
        const EndCriteria& endCriteria,
        OptimizationMethod& method);
\end{minted}
\end{frame}

\section{Specific Plan for PFE / CVA}

\begin{frame}[fragile]
\frametitle{Adding non exercise nodes}
One could add nodes where no exercise is possible in order to be able to approximate
the continuation NPV by regression for PFE, CVA simulations. For this we can just
set the \verb+exerciseValue+ to $-\infty$ at the corresponding nodes.
\end{frame}

\begin{frame}
\frametitle{Gaussian1dMcSwaptionEngine}
Can be copied from Gaussian1dSwaptionEngine and then transformed to a MC engine,
providing an additional method to approximate NPVs by regression for usage in PFE,
CVA simulations.
Working on the generic model process, it can then be used for Hull White 1F and
Markov Functional without any further effort.
\end{frame}

\section{Questions}

\frame{\frametitle{Thank you,  Questions ?}
\begin{figure}
	\centering
		\includegraphics{/home/peter/Pictures/Beaker2.png}
\end{figure}
}



\end{document}
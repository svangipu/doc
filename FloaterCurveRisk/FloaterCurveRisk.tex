%% Based on a TeXnicCenter-Template by Gyorgy SZEIDL.
%%%%%%%%%%%%%%%%%%%%%%%%%%%%%%%%%%%%%%%%%%%%%%%%%%%%%%%%%%%%%

%------------------------------------------------------------
%
\documentclass{amsart}
%
%----------------------------------------------------------
% This is a sample document for the AMS LaTeX Article Class
% Class options
%        -- Point size:  8pt, 9pt, 10pt (default), 11pt, 12pt
%        -- Paper size:  letterpaper(default), a4paper
%        -- Orientation: portrait(default), landscape
%        -- Print size:  oneside, twoside(default)
%        -- Quality:     final(default), draft
%        -- Title page:  notitlepage, titlepage(default)
%        -- Start chapter on left:
%                        openright(default), openany
%        -- Columns:     onecolumn(default), twocolumn
%        -- Omit extra math features:
%                        nomath
%        -- AMSfonts:    noamsfonts
%        -- PSAMSFonts  (fewer AMSfonts sizes):
%                        psamsfonts
%        -- Equation numbering:
%                        leqno(default), reqno (equation numbers are on the right side)
%        -- Equation centering:
%                        centertags(default), tbtags
%        -- Displayed equations (centered is the default):
%                        fleqn (equations start at the same distance from the right side)
%        -- Electronic journal:
%                        e-only
%------------------------------------------------------------
% For instance the command
%          \documentclass[a4paper,12pt,reqno]{amsart}
% ensures that the paper size is a4, fonts are typeset at the size 12p
% and the equation numbers are on the right side
%
\usepackage{amsmath}%
\usepackage{amsfonts}%
\usepackage{amssymb}%
\usepackage{graphicx}
\usepackage[miktex]{gnuplottex}
\usepackage{epstopdf}
%------------------------------------------------------------
% Theorem like environments
%
\newtheorem{theorem}{Theorem}
\theoremstyle{plain}
\newtheorem{acknowledgement}{Acknowledgement}
\newtheorem{algorithm}{Algorithm}
\newtheorem{axiom}{Axiom}
\newtheorem{case}{Case}
\newtheorem{claim}{Claim}
\newtheorem{conclusion}{Conclusion}
\newtheorem{condition}{Condition}
\newtheorem{conjecture}{Conjecture}
\newtheorem{corollary}{Corollary}
\newtheorem{criterion}{Criterion}
\newtheorem{definition}{Definition}
\newtheorem{example}{Example}
\newtheorem{exercise}{Exercise}
\newtheorem{lemma}{Lemma}
\newtheorem{notation}{Notation}
\newtheorem{problem}{Problem}
\newtheorem{proposition}{Proposition}
\newtheorem{remark}{Remark}
\newtheorem{solution}{Solution}
\newtheorem{summary}{Summary}
\numberwithin{equation}{section}
%--------------------------------------------------------

\begin{document}
\title[Risky Floater Curve Risk]{Risky Floater Curve Risk}
\author{Peter Caspers}
\email[Peter Caspers]{pcaspers1973@gmail.com}
\date{February 16, 2013}
\dedicatory{First Version February 16, 2013 - This Version February 16, 2013}
\begin{abstract}
It is well known that a plain riskless floater is worth par and has zero interest rate delta immediately before a fixing in a classic one curve setup. We investigate the structure of the delta under credit risk, a first fixing and a margin added to the payoff. We decompose the delta into three parts representing the different sources of risk.
\end{abstract}

\maketitle

\section{Introduction}

We consider a plain vanilla floating rate note paying an index (like Euribor 6m, USD Libor 3m) plus possibly a spread $m$. The valuation of the note
is done by estimating the float coupons on a curve $C$ and by discounting the cashflows on the same curve plus a (termstructure of) spread $s$. Note that this
setup is quite general because the spread effectively allows for an arbitrary distinction between discounting and forwarding curve as
required in a modern curve setup.

We are interested in the DV01, i.e. the change in the npv $\nu$ of the note when the curve $C$ is shifted in parallel by 1 basis point.
Technically we will work with the derivative of $\nu$ by continuous zero rates on the relevant maturities of the note. We call this derivative
delta. In particular (different from DV01) delta is normalized to the unit.

For the ease of presentation we make some minor simplifying assumptions, namely that there is no fixing delay, i.e. the fixing date is equal to the
start date of an interest rate period and that each index estimation period is equal to the corresponding interest rate period. 

We start with a simple special case before moving to the general case.

\section{No margin, no first fixing}

We assume a flat coupon, i.e. $m=0$, and the valuation time $0$ to be immediately before the first coupon fixing. We then have a time grid $\{t_0, t_1, ... , t_n\}$ with fixing times $0+ = t_0 < t_1 < ... < t_{n-1}$, accrual periods $\tau_i = t_i - t_{i-1}$ for $i=1,...,n$ and payment times $t_1, ... , t_n$. The npv of the floater is

\begin{equation}
\nu = \sum_{i=1}^n \tau_i \left( \frac{P(0,t_{i-1})-P(0,t_i)}{\tau_i P(0,t_i)} \right) P^D(0,t_i) + P^D(0,t_n)
\end{equation}

with $P$ denoting the discount factor on curve $C$ and $P_D$ the discount factor on the spreaded curve $C+s$. Note that the notional is normalized to $1$ here and henceforth. Using continuous zero rates we can write

\begin{equation}\label{npva}
\nu = \sum_{i=1}^n (e^{-r_{i-1} t_{i-1}} - e^{-r_i t_i}) e^{-s_i t_i} + e^{-(r_n+s_n) t_n}
\end{equation}

with $r_i = r(t_i)$ the zero rate for maturity $t_i$ and likewise $s_i=s(t_i)$ the spread to be applied on maturity $t_i$.  We assume non negative spreads for the following derivations, but the results can be adapted for negative spreads in a straightforward way. 

The parallel delta i.e. the delta corresponding to a parallel shift of the curve is given by

\begin{equation}\label{dv01a}
\frac{\partial \nu}{\partial r} =  \sum_{i=1}^n ( t_i e^{- r_i t_i} - t_{i-1} e^{- r_{i-1} t_{i-1}}) e^{-s_i t_i} - t_n e^{-(r_n+s_n) t_n}   
\end{equation}

Here $\partial / \partial r$ denotes the directional derivative along the vector $(1,...,1)$. Bucket deltas can be computed likewise as partial derivatives $\partial / \partial r_j$ 

\begin{equation}\label{dv01b}
\frac{\partial \nu}{\partial r_j} =  t_je^{-r_jt_j} ( e^{-s_jt_j} - e^{-s_{j+1} t_{j+1}})
\end{equation}

for $j = 0,...,n-1$. For $j=0$ and $j=n$ we get a partial derivative of zero. Of course by direct comparision of \ref{dv01a} and \ref{dv01b} or more generally by applying the basic calculus of total derivatives we have

\begin{equation}
\sum_{j=1}^{n-1} \frac{\partial \nu}{\partial r_j} = \frac{\partial \nu}{\partial r}
\end{equation}

From \ref{npva} we see directly that in the case of $s_i \equiv 0$ the npv of the floater is exactly par $=1$. From \ref{dv01a} and \ref{dv01b} on the other hand it is clear that the parallel delta as well as all bucket deltas are zero in this limit case. It it also clear that if $s_i \rightarrow \infty$ for all $i$ the parallel delta and the bucket deltas will converge to $0$.

We shall impose an additional restriction on $s_i$ from now on, namely that

\begin{equation}\label{afs}
e^{-s_i t_i} \leq e^{-s_{i-1} t_{i-1}}
\end{equation}

If the $s_i$ represent pure credit spreads this is just the condition for an arbitrage free credit spread term structure. If (a portion of) the $s_i$ represent the basis between the benchmark discounting curve and forwarding curve, \ref{afs} becomes a real restriction on the basis / credit spread termstructure. If \ref{afs} is fulfilled from \ref{dv01a} we get that

\begin{equation}
\frac{\partial \nu}{\partial r} \geq  \sum_{i=1}^n ( t_i e^{- (r_i+s_i) t_i} - t_{i-1} e^{- (r_{i-1}+s_{i-1}) t_{i-1}}) - t_n e^{-(r_n+s_n) t_n}  = 0
\end{equation}

i.e. the parallel delta is non negative. For the bucket deltas we see from \ref{dv01b} that $\partial \nu / \partial r_j$ is non negative for $j=1,...,n-1$.

Summarizing we see that delta is zero at zero and infinite spread and non negative in between. Figure \ref{deltaFig} shows an example of the shape of this dependency.

\begin{figure}[htbp]
\caption{delta of a 5y note with semiannual payments, no margin, no first fixing, delta is plotted for different spread levels from 0\% to 1000\%, yield term structure is flat @3\%, spread is flat}
\label{deltaFig}
	\begin{gnuplot}
		set terminal epslatex color
		set xrange [0:10.0]
		set yrange [0:1.5]
		unset key
		set xlabel "spread s"
		set ylabel "delta"
		plot 'spreadex.txt' with line
	\end{gnuplot}
\end{figure}

\section{First Fixing and/or non zero margin}

Next we generalize the setup allowing for a first fixing making the first coupon a fixed coupon and thus no longer estimated on the rate curve $C$. We also include the case of non zero, positive margins. The resulting cashflows of the note can be split into a fixed part

\begin{equation}
\nu_f = \sum_{i=1}^n \alpha_i \tau_i e^{-(r_i+s_i) t_i}
\end{equation}

with $\alpha_1 = m + f$, the sum of the margin $m$ and the index fixing $f$ (we assume both non negative), and $\alpha_i = m$ for $i\geq2$, and a float part (including the terminal capital payment)

\begin{equation}
\nu_e = \sum_{i=2}^n (e^{-r_{i-1} t_{i-1}} - e^{-r_i t_i}) e^{-s_i t_i} + e^{-(r_n+s_n) t_n}
\end{equation}

To make use of the results in the previous section we set $t'_i := t_i-t_1$ for $i\geq2$ and write

\begin{equation}
\nu_e = e^{-(r_1+s_1) t_1} \left( \sum_{i=2}^n (e^{-r'_{i-1} t'_{i-1}} - e^{-r'_i t'_i}) e^{-s'_i t'_i} + e^{-(r'_n+s'_n) t'_n} \right)
\end{equation}

with forward rates $r'_i$ and forward spreads $s'_i$ defined by the relations

\begin{eqnarray}
e^{-r'_i (t_i-t_1)}e^{-r_1 t_1} = e^{-r_i t_i}  \\
e^{-s'_i (t_i-t_1)}e^{-s_1 t_1} = e^{-s_i t_i}
\end{eqnarray}

The npv $\nu$ of the note is obviously given by the sum

\begin{equation}
\nu = \nu_f + \nu_e
\end{equation}

The (parallel) delta of the fixed part is easily derived as

\begin{equation}\label{dv01f}
\frac{\partial \nu_f}{\partial r} = -\sum_{i=1}^{n} t_i \alpha_i \tau_i e^{-(r_i+s_i) t_i}
\end{equation}

Clearly, this is a negative number always which tends to zero if $s_i\rightarrow\infty$. It is also clear that the single terms in the sum constitute the
partial derivatives, i.e. the bucket deltas. We call this part of the delta the fixed delta.

The delta of the float part is given by

\begin{equation}\label{dv01e}
\frac{\partial \nu_e}{\partial r} = -t_1 e^{-(r_1+s_1)t_1} \nu'_e +  e^{-(r_1+s_1) t_1} \Delta_e
\end{equation}

with $\nu'_e$ denoting the $t_1$-forward npv of the float part of the note, i.e.

\begin{equation}
\nu'_e = \left( \sum_{i=2}^n (e^{-r'_{i-1} t'_{i-1}} - e^{-r'_i t'_i}) e^{-s'_i t'_i} + e^{-(r'_n+s'_n) t'_n} \right)
\end{equation}

and $\Delta_e$ being the $t_1$-forward delta, i.e. the derivative of the $t_1$-forward npv by parallel shifts of the curve (cf. \ref{dv01a}):

\begin{equation}
\Delta_e =  \sum_{i=2}^n ( t'_i e^{- r'_i t'_i} - t'_{i-1} e^{- r'_{i-1} t'_{i-1}}) e^{-s'_i t'_i} - t'_n e^{-(r'_n+s'_n) t'_n} 
\end{equation}

The delta of the float part therefore decomposes into a negative part represented by the first term in \ref{dv01e} which we call floater discounting delta
and into a positive part which was investigated in the previous section represented by the second term in \ref{dv01e} which we call pure floater delta.

The bucket deltas of the float part are given as follows

\begin{equation}\label{dv01g}
\frac{\partial \nu_e}{\partial r_j} = -t_1 e^{-(r_1+s_1)t_1} \nu'_e \delta_{i1} +  e^{-(r_1+s_1) t_1} \Delta_e^j
\end{equation}

with $\delta_{i1} = 1$ for $i=1$ and zero otherwise and $\Delta_e^j$ analogous to $\ref{dv01b}$ 

\begin{equation}\label{dv01h}
\Delta_e^j =  t'_je^{-r'_jt'_j} ( e^{-s'_jt'_j} - e^{-s'_{j+1} t'_{j+1}})
\end{equation}

for $j=0,...,n-1$ and $\Delta_e^n = 0$. That means that the discounting delta is on bucket $1$ always while the pure floater delta spreads over all buckets except $n$ where it is zero.

Whether delta is positive or negative is dependent on the deal structure, the market data and in particular on the credit spread. Figure \ref{deltaFig2} shows an example including a first fixing and a margin. The parallel delta is negative for small spreads $s$ and gets positive as soon as the fixed part delta and the floater discounting delta is supercompensated by the pure floater delta, which in the example happens around $s=5\%$.

\begin{figure}[htbp]
\caption{delta of a 5y note with semiannual payments, 120bp margin, first fixing 3.2\% at time t=-0.25, delta (pink) is decomposed into fixed part (red), floater discounting part (green) and pure floater part (blue), plotted for different spread levels from 0\% to 100\%, yield term structure is flat @3\%, spread is flat}
\label{deltaFig2}
	\begin{gnuplot}
		set terminal epslatex color
		set xrange [0:1.0]
		set yrange [-0.5:1.5]
		unset key
		set xlabel "spread s"
		set ylabel "delta"
		plot 'spreadex2.txt' using 1:2 with line, 'spreadex2.txt' using 1:3 with line, 'spreadex2.txt' using 1:4 with line, 'spreadex2.txt' using 1:5 with line
	\end{gnuplot}
\end{figure}

\end{document}
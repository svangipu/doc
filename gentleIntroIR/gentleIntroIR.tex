\documentclass{beamer}

\usetheme{CambridgeUS}
\usefonttheme{professionalfonts}

\usepackage{graphicx}
\usepackage[miktex]{gnuplottex}
\ShellEscapetrue
\usepackage{epstopdf}
\usepackage{minted}

\usemintedstyle{manni}
\definecolor{mintedBg}{rgb}{0.98,0.98,0.70}

\begin{document}
\title{A gentle introduction to IR modeling}  
\author{Peter Caspers}
\institute{IKB}
\date{May 23, 2015} 

\frame{\titlepage} 

\frame{\frametitle{Table of contents}\tiny\tableofcontents[hideallsubsections]} 

\section{Introduction}

\frame{\frametitle{What it is all about}
\begin{center}
\Huge Price a derivative contract !
\end{center}
}

\frame{\frametitle{Constraints}
\begin{center}
Do that consistently with all observable relevant market prices.
\end{center}
}

\frame{\frametitle{Second Thoughts}
What do you mean by ``price'' after all ?
\begin{enumerate}
\item the amount of money that can be realized in the market
\item the amount of money that we would require to receive or pay to close the deal
\end{enumerate}
... we will focus on the first kind of price.
}

\frame{\frametitle{Price Adjustments}
\begin{enumerate}
\item CVA
\item DVA
\item FVA
\item KVA
\item MVA
\item TVA
\end{enumerate}
... we exclude all of them in the following.
}

\section{Machinery}

\frame{\frametitle{Randomness}
We need a source of randomness to model the market which is (or appears to be) random.
}

\frame{\frametitle{Stochastic Processes}
To model a random observable we assume $X(0)=x_0$ and

\begin{equation}
dX(t,\omega) = \mu(t,\omega) dt + \sigma(t,\omega) dW
\end{equation}

which means, knowing $X(t_0)$ we can evolve forward by

\begin{equation}
X(t_0+\Delta t) \approx X(t_0) + \mu(t) \Delta t + \sigma(t) Z
\end{equation}

with $Z\sim N(0,\Delta t)$ a normal distributed random variable with zero mean and variance $\Delta t$.
}

\section{Classification}

\frame{\frametitle{Asset Classes}
Usually we distinguish
\begin{enumerate}
\item Equity
\item FX
\item Interest Rates
\item Credit
\item Inflation
\item Commodity
\end{enumerate}
Today we focus on Interest Rates.
}

\frame{\frametitle{Interest Rate Building Blocks}
Quoted, liquidly traded instruments:
\begin{enumerate}
\item Cash deposits
\item Forward Rate Agreements
\item Vanilla Swaps
\item Short Futures
\item Bond Futures
\end{enumerate}
and in addition options
\begin{enumerate}
\item (Ibor) Caps, Floors
\item Swaptions
\item Futures options
\end{enumerate}
}

\frame{\frametitle{Market Models for Options}
Prices for Caps, Floors and Swaptions are quoted using volatilities $\sigma$, assuming a black model for the underlying $F$, either in lognormal style
\begin{equation}
dF = F \sigma dW
\end{equation}
or shifted lognormal style
\begin{equation}
dF = (F+\alpha) \sigma dW
\end{equation}
or in normal style
\begin{equation}
dF = \sigma dW
\end{equation}
all of them having a simple closed form solution (Black formula).
}

\frame{\frametitle{Volatility smile}
Options with different strike and maturity are priced using different volatilities. This seems weird at first sight, but simply means that the Black model is more a quoting convention than a global model.
}

\frame{\frametitle{Underlying density}
It is simple to derive the underlying density as
\begin{equation}
\phi(K) = \frac{\partial^2 c}{\partial K^2} \left(K\right)
\end{equation}
for undiscounted call prices. The density is in the natural pricing measure (T-forward measure for caplets, Annuity measure for swaptions). This means you can get (in principle) the market density for an expiry time from a continuum of quoted option prices in strike direction.
}

\frame{\frametitle{IR Smile modeling}
In interest rates by far the most common single maturity smile model is the (shifted) SABR model
\begin{eqnarray}
dF(t) &=& (F(t)+\alpha)^\beta \sigma(t) dW \\
d\sigma(t) &=& \sigma(t) \nu dV \\
dV dW &=& \rho dt
\end{eqnarray}
with parameters $\beta$ (skew), $\alpha=\sigma(0)$ (initial level for $\sigma$), $\nu$ (volatility of volatility) and $\rho$ (correlation between forward and vol process).
}


\frame{\frametitle{Model independent pricing}
Coupons of the following type can be priced only using a no arbitrage argument on today's yield curves:
\begin{enumerate}
\item fixed coupons
\item natural floating rate coupons
\end{enumerate}
Natural means an IBOR - styled coupon with payment date = index accrual period end.
}

\frame{\frametitle{Single maturity model pricing}
Coupons that depend on the distribution of an underlying variable on a single time instance, like
\begin{enumerate}
\item in arrears fixed IBOR coupons
\item capped / floored IBOR coupons
\item CMS coupons
\item CMS Spread Coupons
\end{enumerate}
Usually the effective rate $r$ for coupon estimation is then written
\begin{equation}
r = r_0 + c
\end{equation}
as the sum of the zero volatility forward (i.e. the forward on today's curve) $r_0$ plus a convexity adjustment $c$.
}

\frame{\frametitle{Closed form convexity adjustments}
Assuming a black model for the underlying one can derivate closed form approximations for convexity adjustments
\begin{enumerate}
\item timing adjustments
\item CMS adjustments
\end{enumerate}
}

\frame{\frametitle{Volatility smile}
Usually options with different strikes are prices

}




\section{Questions}

\begin{frame}[fragile]
\frametitle{Questions / Discussion}
\resizebox{\textwidth}{!}{
\begin{minipage}{0.75\textwidth}
\begin{figure}
	\centering
		\includegraphics{../../../Pictures/erlkoenige/french_70s_automodule05.jpg}
\end{figure}
\end{minipage}}
\\
\begin{center}
French Erlk\"onig
\end{center}
\end{frame}
\end{document}


%% Based on a TeXnicCenter-Template by Gyorgy SZEIDL.
%%%%%%%%%%%%%%%%%%%%%%%%%%%%%%%%%%%%%%%%%%%%%%%%%%%%%%%%%%%%%

%------------------------------------------------------------
%
\documentclass{amsart}
%
%----------------------------------------------------------
% This is a sample document for the AMS LaTeX Article Class
% Class options
%        -- Point size:  8pt, 9pt, 10pt (default), 11pt, 12pt
%        -- Paper size:  letterpaper(default), a4paper
%        -- Orientation: portrait(default), landscape
%        -- Print size:  oneside, twoside(default)
%        -- Quality:     final(default), draft
%        -- Title page:  notitlepage, titlepage(default)
%        -- Start chapter on left:
%                        openright(default), openany
%        -- Columns:     onecolumn(default), twocolumn
%        -- Omit extra math features:
%                        nomath
%        -- AMSfonts:    noamsfonts
%        -- PSAMSFonts  (fewer AMSfonts sizes):
%                        psamsfonts
%        -- Equation numbering:
%                        leqno(default), reqno (equation numbers are on the right side)
%        -- Equation centering:
%                        centertags(default), tbtags
%        -- Displayed equations (centered is the default):
%                        fleqn (equations start at the same distance from the right side)
%        -- Electronic journal:
%                        e-only
%------------------------------------------------------------
% For instance the command
%          \documentclass[a4paper,12pt,reqno]{amsart}
% ensures that the paper size is a4, fonts are typeset at the size 12p
% and the equation numbers are on the right side
%
\usepackage{amsmath}%
\usepackage{amsfonts}%
\usepackage{amssymb}%
\usepackage{graphicx}
%------------------------------------------------------------
% Theorem like environments
%
\newtheorem{theorem}{Theorem}
\theoremstyle{plain}
\newtheorem{acknowledgement}{Acknowledgement}
\newtheorem{algorithm}{Algorithm}
\newtheorem{axiom}{Axiom}
\newtheorem{case}{Case}
\newtheorem{claim}{Claim}
\newtheorem{conclusion}{Conclusion}
\newtheorem{condition}{Condition}
\newtheorem{conjecture}{Conjecture}
\newtheorem{corollary}{Corollary}
\newtheorem{criterion}{Criterion}
\newtheorem{definition}{Definition}
\newtheorem{example}{Example}
\newtheorem{exercise}{Exercise}
\newtheorem{lemma}{Lemma}
\newtheorem{notation}{Notation}
\newtheorem{problem}{Problem}
\newtheorem{proposition}{Proposition}
\newtheorem{remark}{Remark}
\newtheorem{solution}{Solution}
\newtheorem{summary}{Summary}
\numberwithin{equation}{section}
%--------------------------------------------------------

\begin{document}
\title[Normal Libor in arrears]{Normal Libor in arrears}
\author{Peter Caspers}
\email[Peter Caspers]{pcaspers1973@gmail.com}
\date{December 10, 2012}
\dedicatory{First Version December 10, 2012 - This Version December 10, 2012}
\begin{abstract}
We derive the in arrears covexity adjustment in a normal black model.
\end{abstract}

\maketitle

\section{Introduction}

Consider a Libor rate $F$ with fixing time $t_f$, index start date $t_s$ and index end date $t_e$ which is paid at $t_p$. We are interested in the valuation of a coupon involving such a Libor rate fixed in arrears, i.e. $t_f = t_p - \Delta$ with $\Delta$ denoting the settlement lag (e.g. 2 business days). For the following derivations we assume $\Delta=0$. We also assume $t_s = t_p =:t, t_e = t_s + \tau$ with $\tau=\tau_{\textnormal{accr}}$ the accrual period of the coupon to be valued. These are typically all minor approximations to the reality.

\section{The model}

We assume a normal black model for $F$, i.e.

\begin{equation}
dF = \nu dW
\end{equation}

with $\nu$ denoting the normal volatility. 

\section{The adjustment}

The Libor coupon can then be valued (working in the $t+\tau$ forward measure) as 

\begin{equation}\label{generaladjustment}
C = P(0,t+\tau) E^{t+\tau} \left( \frac{F(t)\tau }{P(t,t+\tau)} \right) 
\end{equation}

with $P(t,T)$ denoting the price at time $t$ of a zero bond with maturity $T$ and the expectation being taken in the $t+\tau$-forward measure $\mathbb{Q}^{t+\tau}$.  We note here that our derivation will be done in a single curve setup for the moment, i.e. with identical forwading and discounting curves.

We write

\begin{equation}
F(t) = \frac{F(t)}{1+F(t)\tau} (1+F(t)\tau)
\end{equation}

and get

\begin{equation}
C = P(0,t+\tau) E^{t+\tau}(\tau F(t) + \tau^2 F^2(t))
\end{equation}

Now $F(\cdot)$ is a $\mathbb{Q}^{t+\tau}$ - martingale. Furhtermore by Ito

\begin{equation}
dF^2 = 2 F dF + \nu dF dF = 2F\nu dW + \nu^2 dt
\end{equation}

hence 

\begin{equation}
E^{t+\tau}(F^2(t)) = F^2(0) + \nu^2 t
\end{equation}

and therefore

\begin{equation}
C = P(0,t+\tau) (\tau F(0) + \tau^2 F^2(0) + \tau^2 \nu^2 t )
\end{equation}

which can be written as 

\begin{equation}
P(0,t) \tau \left( F(0) + \frac{\tau \nu^2 t}{1+\tau F(0)}\right)
\end{equation}

since $P(0,t+\tau) = P(0,t) (1+\tau F(0))$. It is interesting to note that this adjustment is the same as the in arrears limit case in (3.14) of \cite{LiborTimingAdjustments} which is derived in a lognormal model if one relates the normal volatility $\nu$ and the lognormal volatility $\sigma$ by $\nu := F(0) \sigma$, i.e. the drift freezing \textit{approximation} in \cite{LiborTimingAdjustments} effectively means to compute the convexity adjustment \textit{exactly} in a normal black model when connecting the parameters of the two models appropriately.

\section{Caplets}

Now we consider an in arrears caplet. Just as above it can be valued

\begin{equation}
C = P(0,t+\tau)E^{t+\tau} \left( \frac{ (F(t)-K)^+ \tau } { P(t, t+\tau)} \right)
\end{equation}

We write

\begin{equation}
(F(t)-K)^+ = \frac{(F(t)-K)^+}{1+F(t)\tau}(1+F(t)\tau)
\end{equation}

to arrive at

\begin{equation}
C=P(0,t+\tau)E^{t+\tau} ( \tau(F(t)-K)^+ + \tau^2 F(t)(F(t)-K)^+ )
\end{equation}

We know from above that $F(t)$ is normally distributed with mean $F(0)$ and standard deviation $\nu\sqrt{t}$. Introducing a new variable

\begin{equation}
X = \frac{F(t)-F(0)}{\nu\sqrt{t}}
\end{equation}

which is obviously standard normal we can evaluate the components from the expectation above as

\begin{equation}
A :=\frac{1}{\sqrt{2\pi}} \int_{(K-F(0))/\nu\sqrt{t}}^\infty (X\nu\sqrt{t}+F(0)-K) e^{-X^2/2} dX
\end{equation}

and

\begin{equation}
B := \frac{1}{\sqrt{2\pi}} \int_{(K-F(0))/\nu\sqrt{t}}^\infty (X\nu\sqrt{t}+F(0))(X\nu\sqrt{t}+F(0)-K) e^{-X^2/2} dX
\end{equation}

These integrals can be computed using the general identity

\begin{equation}
\frac{1}{\sqrt{2\pi}} \int (aX^2+bX+c)e^{-X^2/2}dX = \frac{a+c}{2} \textnormal{erf}\left(\frac{X}{\sqrt{2}}\right) - \frac{1}{\sqrt{2\pi}}e^{-X^2/2}(aX+b)
\end{equation}

and putting all together we retrieve the caplet price as

\begin{equation}
C=P(0,t+\tau) ( \tau A + \tau^2 B )
\end{equation}


\begin{thebibliography}{2}

\bibitem {LiborTimingAdjustments}Caspers Peter, \textit{Libor Timing Adjustments},
Electronic copy available at: http://ssrn.com/abstract=2170721

\end{thebibliography}

\end{document}
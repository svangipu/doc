\documentclass{beamer}

\usetheme{CambridgeUS}
\usefonttheme{professionalfonts}

\begin{document}
\title{GSR Development Overview}  
\author{Peter Caspers}
\date{April 6, 2013} 

\frame{\titlepage} 

\frame{\frametitle{Table of contents}\tableofcontents[hideallsubsections]} 

\section{Model}

\frame{\frametitle{Model Spec}

\begin{itemize}
\item GSR Model = Gaussian Short Rate Model = Hull White Model
\item 1 factor
\item time dependent deterministic volatility and reversion
\item dynamics is in T-Forward measure
\item formulation without "fitting function"
\end{itemize}
}

\frame{\frametitle{Model Class Gsr}

\begin{itemize}
\item constructors for fixed mean reversion, calibrated mean reversion
\item sugar constructor for constant mean reversion
\item state process class GsrProcess with exact formulas
\item caching necessary because of fomula complexity
\end{itemize}

}

\section{Instruments}

\frame{\frametitle{Instrument Class NonStandardSwap}

\begin{itemize}
\item fixed float swap with periodic notionals and / or fixed rates
\item direct generalization of Vanilla Swap
\item copy constructor from Vanilla Swap
\end{itemize}

}

\frame{\frametitle{Instrument Class NonStandardSwaption}

\begin{itemize}
\item option on NonstandardSwap
\item just like Swaption is an option on Vanilla Swap
\item copy constructor from Swaption
\item physical delivery
\item european or bermudan exercise
\end{itemize}

}

\section{Pricing Engines}

\frame{\frametitle{Engine Class GsrSwaptionEngine}

\begin{itemize}
\item pricing engine for european and bermudan swaptions
\item static multicurve support
\item uses numerical integration
\item payoff is interpolated with Lagrange Splines
\item exact integration of splines against gaussian distribution
\item coverage of n (default 7) standard deviations of the state variable with m (default 64) grid points
\item backward engine for bermudan exercise rights
\end{itemize}

}

\frame{\frametitle{Engine Class GsrJamshidianSwaptionEngine}

\begin{itemize}
\item pricing engine for european swaptions 
\item uses the Jamshidian trick
\item exact pricing 
\item monocurve only
\item extended version of the ql implementation respecting the start delay of the underlying swap
\item integral engine is usually just as good (accuracy, speed)
\end{itemize}

}

\frame{\frametitle{Engine Class GsrNonstandardSwaptionEngine}

\begin{itemize}
\item pricing engine for european and bermudan swaptions with periodic notional and / or fixed rate 
\item static multicurve support, uses numerical integration, just as GsrSwaptionEngine
\item provides generation of suitable calibration baskets (method see below), baskets can be retrieved from the instrument
\end{itemize}

}

\section{QuantLib extensions}

\frame{\frametitle{SwaptionHelper}

\begin{itemize}
\item provide non atm calibration instruments (ensuring that they are otm)
\item provide nominal $\neq 1$ (meaningless for calibration, but useful information for non standard calibration baskets)
\end{itemize}

}

\frame{\frametitle{Iterative model calibration}

\begin{itemize}
\item provide an iterative model calibration in addition to the global optimization
\item for boostrapping the piecewise volatility function in the GSR model
\item much faster than global calibration if many volatility step dates are present ($> 10$)
\item of course not generally applicable, so handle with care (similar situation as with iterative and global bootstrapper for yield curves)
\end{itemize}

}


\section{Nonstandard Swaption Calibration}

\frame{\frametitle{Representative Swaption Approach}

Problem: Find a standard (market quoted up to interpolation in option maturity or swap length) swaption which is "similar" to a non standard european
swaption.

\begin{itemize}
\item "Similar" should mean: the underlyings on which the options are written should at option expiry have close npvs in all states of the world. If this is the
case the payoff from both options will be close, too, hence todays npv of both options thereby inducing the npv of the non standard option thanks to the
market value of the quoted option.

\item Obviously "in all states of the world" is a model dependent concept. Here we use the GSR model which use for the non standard bermudan swaption
pricing later on.
\end{itemize}
}

\frame{\frametitle{European non standard swaptions}

\begin{itemize}
\item obviously the approach can also be used to price non standard european swaptions
\item note that the model choice implies perfect correlation of rates 
\item other models are possible ...
\end{itemize}

}

\frame{\frametitle{Procedure to find the representative swaption}

\begin{itemize}
\item the underlying has $3$ free parameters: nominal, fixed rate, maturity
\item we determine these parameters by minimizing the difference in npv, delta and gamma of the two underlyings 
\item delta and gamma should be understood as first and second order derivatives of the npv by the state variable of the model
\item npv, delta and gamma are evaluated at the expectation of the state variable at option expiry
\end{itemize}

}

\frame{\frametitle{Details of the procedure}

\begin{itemize}
\item the minimization is done using numerical optimization (Levenberg Marquardt) 
\item swap maturity is discrete in nature (we use whole months as unit). Since optimization (and in particular Levenberg Marquardt) relies on continuous / differentiable target functions we interpolate npvs between these "physical" swaps
\item furthermore to give the optimizer a nice target function, negative nominals and negative fixed rate are interpreted as payer / receiver swaps 
\item delta and gamma are computed by finite differences and expressed as derivatives in the normalized (zero mean, unit variance) state variable
\item npv and delta are normalized by the target delta, gamma by the target gamma
\end{itemize}

}

\frame{\frametitle{A fixed point problem ?}

\begin{itemize}
\item the npv, delta and gamma depend on the model parameters (sigma up to expiry, reversion up to forward measure time)
\item therefore after calibration of the model to the basket, the basket changes !
\item should we iterate to find some fixed point ?
\item no, because the dependency of the basket on the model parameters is weak enough to determine the basket on reasonable model
parameters and then calibrating the model the that basket. full stop.
\end{itemize}

}


\frame{\frametitle{Calibration Baskets: Demo / Test cases}

\begin{itemize}
\item standard swaption 
\item decreasing notional (amortizer)
\item increasing notional (accreter)
\item step up coupon
\item step down coupon
\item decreasing notional and step up coupon
\end{itemize}

}


\frame{\frametitle{Robustness of the representation: Demo}

\begin{itemize}
\item match a standard swap 
\item plot the npv of the non standard and the standard swap in 5 standard deviations of the state variable
\item price the standard swaption and the non standard swaption
\end{itemize}

}

\section{Summary}

\frame{\frametitle{Summary}

\begin{itemize}
\item Gsr model class provides time dependent volatility and reversion 
\item pricing engines provide static multi curve support
\item extended non atm swaption calibration helper
\item iterative model calibration in addition to global calibration
\item generation of calibration baskets for bermudan non standard swaptions
\end{itemize}

}



\end{document}

